% Theorem System
% The following boxes are provided:
%   Определение:     \defn 
%   Предупреждение: \asum
%   Теорема:        \thm 
%   Лемма:          \lem
%   Следствие:      \cor
%   Свойство:    \prop   
%   Требование:          \clm
%   Утверждение:           \fact
%   Пример:        \ex
%   Замечание:         \rmk (sentence), \rmkb (block)
% Suffix
%   r:              Allow Theorem/Definition to be referenced, e.g. thmr
%   p:              Add a short proof block for Lemma, Corollary, Proposition or Claim, e.g. lemp
%                   For theorems, use \pf for proof blocks

% ======= Real examples : 

% \defn{Definition Name}{
%     A defintion.
% }

% \asum{Assumption Name}{
%     An assumption.
% }

% \thmr{Theorem Name}{mybigthm}{
%     A theorem.
% }

% \lem{Lemma Name}{
%     A lemma.
% }

% \fact{
%     A fact.
% }

% \cor{
%     A corollary.
% }

% \prop{
%     A proposition.
% }

% \pf{
%    proof.
% }

% \exm{
%     some examples. 
% }

% \rmk{
%     Some remark.
% }

% \rmkb{
%     Some more remark.
% }

\usepackage{xcolor}

% Синие и голубые оттенки
\definecolor{defscol}{RGB}{41, 128, 185}   % Определение (темно-синий)
\definecolor{asumscol}{RGB}{135, 206, 250} % Предположение (светло-голубой)
\definecolor{rmkscol}{RGB}{75, 118, 255}   % Замечание (насыщенный синий)

% Голубые оттенки
\definecolor{exmscol}{RGB}{64, 224, 208}   % Пример (светло-голубой)
\definecolor{lemscol}{RGB}{39, 78, 192}    % Лемма (темно-синий)

% Темно-синие оттенки
\definecolor{thmscol}{RGB}{34, 67, 182}    % Теорема (очень темно-синий)
\definecolor{prpscol}{RGB}{113, 158, 209}  % Свойство (умеренно синий)

% Светлые голубые оттенки
\definecolor{corscol}{RGB}{167, 218, 239}  % Следствие (светло-голубой)
\definecolor{clmscol}{RGB}{84, 136, 203}   % Требование (умеренно темно-синий)

% Нейтральный оттенок
\definecolor{facscol}{RGB}{124, 166, 199}  % Утверждение (серо-голубой)

% ============================
% Определение
% ============================
\newtcbtheorem[number within=section]{mydefinition}{Определение}
{
    enhanced,
    frame hidden,
    titlerule=0mm,
    toptitle=1mm,
    bottomtitle=1mm,
    fonttitle=\bfseries\large,
    coltitle=black,
    colbacktitle=defscol!40!white,
    colback=defscol!20!white,
}{defn}

\NewDocumentCommand{\defn}{m+m}{
    \begin{mydefinition}{#1}{}
        #2
    \end{mydefinition}
}

\NewDocumentCommand{\defnr}{mm+m}{
    \begin{mydefinition}{#1}{#2}
        #3
    \end{mydefinition}
}

% ============================
% Предупреждение
% ============================
\newtcbtheorem[use counter from=mydefinition]{myassumption}{Предупреждение}
{
    enhanced,
    frame hidden,
    titlerule=0mm,
    toptitle=1mm,
    bottomtitle=1mm,
    fonttitle=\bfseries\large,
    coltitle=black,
    colbacktitle=asumscol!40!white,
    colback=asumscol!20!white,
}{asum}

\NewDocumentCommand{\asum}{m+m}{
    \begin{myassumption}{#1}{}
        #2
    \end{myassumption}
}

\NewDocumentCommand{\asumr}{mm+m}{
    \begin{myassumption}{#1}{#2}
        #3
    \end{myassumption}
}

% ============================
% Теорема
% ============================

\newtcbtheorem[use counter from=mydefinition]{mytheorem}{Теорема}
{
    enhanced,
    frame hidden,
    titlerule=0mm,
    toptitle=1mm,
    bottomtitle=1mm,
    fonttitle=\bfseries\large,
    coltitle=black,
    colbacktitle=thmscol!40!white,
    colback=thmscol!20!white,
}{thm}

\NewDocumentCommand{\thm}{m+m}{
    \begin{mytheorem}{#1}{}
        #2
    \end{mytheorem}
}

\NewDocumentCommand{\thmr}{mm+m}{
    \begin{mytheorem}{#1}{#2}
        #3
    \end{mytheorem}
}

\newenvironment{thmpf}{
	{\noindent{\it \textbf{Доказательство:}}}
	\tcolorbox[blanker,breakable,left=5mm,parbox=false,
    before upper={\parindent15pt},
    after skip=10pt,
	borderline west={1mm}{0pt}{thmscol!40!white}]
}{
    \textcolor{thmscol!40!white}{\hbox{}\nobreak\hfill$\blacksquare$} 
    \endtcolorbox
}

\NewDocumentCommand{\thmp}{m+m+m}{
    \begin{mytheorem}{#1}{}
        #2
    \end{mytheorem}

    \begin{thmpf}
        #3
    \end{thmpf}
}

% ============================
% Лемма
% ============================

\newtcbtheorem[use counter from=mydefinition]{mylemma}{Лемма}
{
    enhanced,
    frame hidden,
    titlerule=0mm,
    toptitle=1mm,
    bottomtitle=1mm,
    fonttitle=\bfseries\large,
    coltitle=black,
    colbacktitle=lemscol!40!white,
    colback=lemscol!20!white,
}{lem}

\NewDocumentCommand{\lem}{m+m}{
    \begin{mylemma}{#1}{}
        #2
    \end{mylemma}
}

\newenvironment{lempf}{
	{\noindent{\it \textbf{Доказательство:}}}
	\tcolorbox[blanker,breakable,left=5mm,parbox=false,
    before upper={\parindent15pt},
    after skip=10pt,
	borderline west={1mm}{0pt}{lemscol!40!white}]
}{
    \textcolor{lemscol!40!white}{\hbox{}\nobreak\hfill$\blacksquare$} 
    \endtcolorbox
}

\NewDocumentCommand{\lemp}{m+m+m}{
    \begin{mylemma}{#1}{}
        #2
    \end{mylemma}

    \begin{lempf}
        #3
    \end{lempf}
}

% ============================
% Следствие
% ============================

\newtcbtheorem[use counter from=mydefinition]{mycorollary}{Следствие}
{
    enhanced,
    frame hidden,
    titlerule=0mm,
    toptitle=1mm,
    bottomtitle=1mm,
    fonttitle=\bfseries\large,
    coltitle=black,
    colbacktitle=corscol!40!white,
    colback=corscol!20!white,
}{cor}

\NewDocumentCommand{\cor}{+m+m}{
    \begin{mycorollary}{#1}{}
        #2
    \end{mycorollary}
}

\newenvironment{corpf}{
	{\noindent{\it \textbf{Доказательство:}}}
	\tcolorbox[blanker,breakable,left=5mm,parbox=false,
    before upper={\parindent15pt},
    after skip=10pt,
	borderline west={1mm}{0pt}{corscol!40!white}]
}{
    \textcolor{corscol!40!white}{\hbox{}\nobreak\hfill$\blacksquare$} 
    \endtcolorbox
}

\NewDocumentCommand{\corp}{+m+m+m}{
    \begin{mycorollary}{#1}{}
        #2
    \end{mycorollary}

    \begin{corpf}
        #3
    \end{corpf}
}

% ============================
% Свойства
% ============================

\newtcbtheorem[use counter from=mydefinition]{myproposition}{Свойства}
{
    enhanced,
    frame hidden,
    titlerule=0mm,
    toptitle=1mm,
    bottomtitle=1mm,
    fonttitle=\bfseries\large,
    coltitle=black,
    colbacktitle=prpscol!40!white,
    colback=prpscol!20!white,
}{prop}

\NewDocumentCommand{\prop}{+m+m}{
    \begin{myproposition}{#1}{}
        #2
    \end{myproposition}
}

\newenvironment{proppf}{
	{\noindent{\it \textbf{Доказательство:}}}
	\tcolorbox[blanker,breakable,left=5mm,parbox=false,
    before upper={\parindent15pt},
    after skip=10pt,
	borderline west={1mm}{0pt}{prpscol!40!white}]
}{
    \textcolor{prpscol!40!white}{\hbox{}\nobreak\hfill$\blacksquare$} 
    \endtcolorbox
}



\NewDocumentCommand{\propp}{+m+m+m}{
    \begin{myproposition}{#1}{}
        #2
    \end{myproposition}
    \begin{proppf}
        #3
    \end{proppf}
}

% ============================
% Требование
% ============================

\newtcbtheorem[use counter from=mydefinition]{myclaim}{Требование}
{
    enhanced,
    frame hidden,
    titlerule=0mm,
    toptitle=1mm,
    bottomtitle=1mm,
    fonttitle=\bfseries\large,
    coltitle=black,
    colbacktitle=clmscol!40!white,
    colback=clmscol!20!white,
}{clm}

\NewDocumentCommand{\clm}{m+m}{
    \begin{myclaim*}{#1}{}
        #2
    \end{myclaim*}
}

\NewDocumentCommand{\clmp}{m+m+m}{
    \begin{myclaim*}{#1}{}
        #2
    \end{myclaim*}

    \begin{clmpf}
        #3
    \end{clmpf}
}

% ============================
% Утверждение
% ============================

\newtcbtheorem[use counter from=mydefinition]{myfact}{Утверждение}
{
    enhanced,
    frame hidden,
    titlerule=0mm,
    toptitle=1mm,
    bottomtitle=1mm,
    fonttitle=\bfseries\large,
    coltitle=black,
    colbacktitle=facscol!40!white,
    colback=facscol!20!white,
}{fact}

\NewDocumentCommand{\fact}{+m}{
    \begin{myfact}{}{}
        #1
    \end{myfact}
}

% ============================
% Пример
% ============================


\newenvironment{myexample}{
    \tcolorbox[blanker,breakable,left=5mm,parbox=false,
    before upper={\parindent15pt},
    after skip=10pt,
	borderline west={1mm}{0pt}{clmscol!40!white}]
}{
    \textcolor{clmscol!40!white}{\hbox{}\nobreak\hfill$\blacksquare$} 
    \endtcolorbox
}

\NewDocumentCommand{\exm}{m+m}{
    \begin{myexample}
	{\noindent{\it \textbf{Пример #1: }}}\\ 
        #2
    \end{myexample}
}


% ============================
% Замечание
% ============================


\NewDocumentCommand{\rmk}{+m}{
    {\it \color{rmkscol!80!white}#1}
}

\newenvironment{remark}{
    \par
    \vspace{5pt}
    \begin{minipage}{\textwidth}
        {\par\noindent{\textbf{Замечание}}}
        \tcolorbox[blanker,breakable,left=5mm,
        before skip=10pt,after skip=10pt,
        borderline west={1mm}{0pt}{rmkscol!20!white}]
}{
        \endtcolorbox
    \end{minipage}
    \vspace{5pt}
}

\NewDocumentCommand{\rmkb}{+m}{
    \begin{remark}
        #1
    \end{remark}
}
