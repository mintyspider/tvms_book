\documentclass[../Main.tex]{subfiles}
\begin{document}
\chapter{Случайные величины}

\section{Определение случайной величины, ее закона и функции распределения}

\subsection{Определение случайной величины}

\rmk{Случайные величины - это такие общие модели для упрощения анализа испытаний, проще говоря.}

\(] \ (\Omega, \ \mathfrak{A}, \ \mathbb{P})\) - вероятностное пространство.

\defn{Случайная величина}{
Отображение множества элементарных исходов на числовую прямую, удовлетворяющее свойству: 
\(\forall x \in \mathbb{R}^1\) \(\{\omega: \xi (\omega) < x \} \in \mathfrak{A}\)
}

Возьмем отображение \(\mathfrak{A}\) на числовую прямую \(\mathbb{R}^1\) (заменим все события точками на числовой прямой):

\(\xi \ -\) кси, случайная величина. Её отображение удовлетворяет следующему условию: \(\forall x \in \mathbb{R}^1\) \(\{\omega: \xi (\omega) < x \} \in \mathfrak{A}\), где \(\omega \ -\) событие.

Нам нужно измеримое множество. Отсюда можно найти вероятность этого события.

\defn{Дискретная случайная величина}{
Множество принимаемых значений с ненулевой вероятностью дискретно (конечно или счетно).
}

\subsection{Определение закона распределения дискретной случайной величины}

\defn{Закон распределения дискретной случайной величины}{
Соответствие между возможными принимаемыми значениями и их вероятностями.
}

Множество \(\xi\{x_i, \ \mathbb{P}_i \}, \ i \in I\), где \(\mathbb{P}_i \ -\) вероятность того, что принимается значение \(\xi = x_i\). \rmk{Эти события несовместны и образуют полную группу событий.}

Свойства вероятностей:
\begin{enumerate}
    \item \(\forall i:\mathbb{P}_i>0\), \rmk{потому что случайная величина гарантированно принимает заданное значение.}
    \item \(\sum \mathbb{P}_i = 1\), \rmk{так как полная группа событий.}
\end{enumerate}

\subsection{Определение функции распределения}

\defn{Функция распределения случайной величины}{
Вероятность того, что случайная величина примет значение, меньшее x.
}

Обозначается \(F_{\xi}(x) = \mathbb{P}(\xi < x)\).

\rmkb{
\(\xi\) строго меньше x, \(F_\xi(x)\) непрерывна слева.

\(\xi\) имеет единственную функцию распределения.
}

\subsection*{Свойства функции распределения:}

\rmkb{
Любая функция, сосредоточенная от \(-\infty\) до \(+\infty\), со значениями, кроме \(-\infty, \ +\infty\), удовлетворяющая свойствам 1, 3-7 (в общем, кроме 2) - функция распространения какой-либо случайной величины.
}

\begin{enumerate}
    \item Функция распределения изменяется от 0 до 1:
    
    \(0 \leq \mathbb{F}_\xi(x) \leq 1\)
    \item Вероятность принадлежности промежутку - разность функций распределения конца и начала промежутка: \(\mathbb{P}(x_1 \leq \xi < x_2) = \mathbb{F}_\xi(x_2) - \mathbb{F_\xi(x_1)}\)

    \textbf{Доказательство:}\(\{ \xi < x_2\} = \{\xi < x_1\}(1) \cup \{x_1 \leq \xi < x_2\}(2)\), события 1 и 2 независимы.
    
    \(\mathbb{P}(\xi < x_2) = \mathbb{P}(\xi < x_1) + \mathbb{P}(x_1 \leq \xi < x_2)\)

    Выразим \(\mathbb{P}(x_1 \leq \xi < x_2) = \mathbb{P}(\xi < x_2) - \mathbb{P}(\xi < x_1)\), а это ничто иное, как \(\mathbb{F}_\xi(x_2) - \mathbb{F}_\xi(x_1)\), ч.т.д.
    
    \item Функция распределения неубывающая, то есть \(x_1 < x_2 \Rightarrow \mathbb{F}_\xi(x_1) \leq \mathbb{F}_\xi (x_2)\)

    \textbf{Доказательство:} \(из 2^\circ\) : \(\mathbb{F}_\xi(x_2)=\mathbb{F}_\xi(x_1) + ((\mathbb{P}(x_1 \leq \xi < x_2)) \geq 0)\)
    \item \(\mathbb{P}(\xi \geq x) = 1 - \mathbb{P}(\xi < x) = 1 - \mathbb{F}_\xi (x)\)

    \textbf{Доказательство:} \((\xi \geq x) = \overline{(\xi < x)}\). Через противоположное событие: 
    \(\mathbb{P}(\xi \geq x) = 1 - \mathbb{P}(\xi < x)\). По определению функции распределения выражение \(= 1 - \mathbb{F}_\xi (x)\).

    \item Значение функции распределения в точке \(+\infty=1 : \underset{{x \rightarrow +\infty}}{lim}{\mathbb{F}_\xi(x)=1}\).
    
    \textbf{Доказательство:} Рассмотрим монотонно возрастающую последовательность: \(x_1 < x_2 < \dots < x_n, \ \underset{n \rightarrow + \infty}{lim}x_n = + \infty\). События \(A_1 = \{ \xi < x_1\},\ A_2 = \{ x_1 \leq \xi < x_2\},\ A_3 = \{x_2 \leq \xi < x_3\}\) и т.д. \Rightarrow \(\{ \xi < x_n \} = \overset{n}{\underset{i=1}{\sum}}A_i\), потому что все \(A_i\) между собой несовместны. \(\mathbb{P}(\xi < x_n) =\overset{n}{\underset{i=1}{\sum}}\mathbb{P}(A_i)\). Возьмем за достоверное событие \(\Omega = \overset{n}{\underset{i=1}{\bigcup}}A_i\). Тогда \(\mathbb{P}(\Omega)=\mathbb{P}(\overset{n}{\underset{i=1}{\bigcup}}A_i) = \mathbb{P}(\underset{n \rightarrow + \infty}{lim}\overset{n}{\underset{i=1}{\bigcup}}A_i)\). По аксиоме счетной аддитивности \(= \underset{n \rightarrow +\infty}{lim} \mathbb{P}(\overset{n}{\underset{i=1}{\bigcup}}A_i) = \underset{n \rightarrow + \infty}{lim}(\xi < x_n)\). Отсюда мораль: если \( \rightarrow +\infty \Rightarrow \mathbb{F}_\xi \rightarrow 1\).

    \item Значение функции распределения в точке \(-\infty=0 : \underset{{x \rightarrow -\infty}}{lim}{\mathbb{F}_\xi(x)=0}\).

    \textbf{Доказательство:} ] \(x_1, \dots, x_n \ -\) бесконечно убывающая числовая последовательность. \(x_1 > x_2 > \dots > x_n, \ \underset{n \rightarrow - \infty}{lim}x_n = - \infty\). \(A_1 = \{ \xi < x_1\},\ A_2 = \{ \xi < x_2\},\ A_3 = \{\xi < x_3\},\ A_n =\{ \xi < x_n \}\). Пересечение \(\overset{\infty}{\underset{i=1}{\bigcap}} A_i = \O\). \rmk{Это означает, что с увеличением n вероятность того, что \(\xi\) окажется меньше всех \(x_n\), стремится к нулю.}

    \(\mathbb{F}_\xi(x_n) = \mathbb{P}(A_n) \Rightarrow \underset{n \rightarrow \infty}{lim}\mathbb{F}_\xi (x_n) = \underset{n \rightarrow \infty}{lim}\mathbb{P}(A_n)=\mathbb{P}(\O)=0\). \(A_1\) содержит \(A_2\) и т.д., так что пересечение \(\overset{n}{\underset{i=1}{\bigcap}}A_i = A_n \Rightarrow \mathbb{P}(\overset{n}{\underset{i=1}{\bigcap}}A_i)=0\) при \(n \rightarrow -\infty\).
    
    \item Функция распределения непрерывна слева, то есть \(\underset{\Delta \rightarrow 0+}{lim}\mathbb{F}_\xi (x-\Delta) = \mathbb{F}_\xi (x)\).

    \textbf{Доказательство:} \(\{\alpha_n\}\ -\) последовательность положительных чисел. \(\underset{n \rightarrow \infty}{lim}\alpha_n = 0, \alpha_i \geq 0, A_n = \{ x-\alpha_n \leq \xi < x\}\). \(A_n\) содержится в \(A_{n-1}: (A_n < A_{n-1}< \dotsc < A_1)\), \(A_1 \ -\) самое большое событие. Пересечение \(\overset{n}{\underset{i=1}{\bigcap}}A_i = A_n \Rightarrow \overset{n}{\underset{i=1}{\bigcap}}A_i = \O\), т.е. \(\underset{n \rightarrow \infty}{lim} \mathbb{P}(A_n) = \underset{n \rightarrow \infty}{lim}\mathbb{P}(\overset{n}{\underset{i=1}{\bigcap}}A_i) = \mathbb{P}(\underset{n \rightarrow \infty}{lim}\overset{n}{\underset{i=1}{\bigcap}}A_i) = \mathbb{P}(\O) = 0\).

    По свойству \(2^\circ\) \(\mathbb{P}(A_n) = \mathbb{F}_\xi (x) - \mathbb{F}_\xi(x-\alpha_n)\).

    \(\underset{n \rightarrow \infty}{lim}\mathbb{P}(A_n) = \underset{n \rightarrow \infty}{lim}(\mathbb{F}_\xi (x) - \mathbb{F}_\xi(x-\alpha_n))\)

    \(0 = \mathbb{F}_\xi(x) - \underset{n \rightarrow \infty}{lim}\mathbb{F}_\xi(x - \alpha_n)\)

    \(\underset{n \rightarrow \infty}{lim} \mathbb{F}_\xi (x - \alpha_n) = \mathbb{F}_\xi (x)\)

    Так как предел равен значению функции, то это предел слева.
\end{enumerate}

\section{Примеры случайных величин}

\subsection*{Пример 1: закон Бернулли}
Говорят, что случайная величина \(\xi\) распределена по \textbf{закону Бернулли}, если \(\mathbb{P}(\xi = 1) = p, \ \mathbb{P}(\xi =  0) = 1-p\).

\subsection*{Пример 2}

\(\xi : \{1, \dots, n\}\), если \(\mathbb{P}(\xi = k) = \dfrac{1}{n}\)

\subsection*{Пример 3: биномиальная случайная величина}

\(\xi:n \in N, n \geq 2 \ (0, 1, 2, \dots, n), n+1\) значений.

\(\mathbb{P}(\xi = k) =\mathbb{P}_k = C_n^k \cdot p^k \cdot (1-p)^{n-k}\) - число успехов в системе Бернулли, т.е. случайная величина задается биномиальным распределением.
\end{document}