\documentclass[../Main.tex]{subfiles}
\begin{document}
\chapter{Дисперсия}

\section{Определение}

\defn{Дисперсия}{
Второй центральный момент.

\[\exists \ \mathbb{D}_\xi = \mathbb{M}[(\xi - \mathbb{M}_\xi)^2] \]
}

\subsection*{Для вычисления на практике}

\(\mathbb{D}_\xi = \mathbb{M}[(\xi - \mathbb{M}_\xi)^2] = \mathbb{M}[(\xi^2 - 2\xi\mathbb{M}_\xi + (\mathbb{M}_\xi)^2)] = \mathbb{M}[\xi^2] - 2(\mathbb{M}_\xi)^2 + (\mathbb{M}_\xi)^2 = \mathbb{M}[\xi^2] - (\mathbb{M}_\xi)^2\)

\rmk{Почему возможно это реализовать? Математическое ожидание обладает свойством линейности, то есть можно суммировать и выносить константы. Также математическое ожидание от константы - константа, в итоге мы получаем разность второго и первого начальных моментов}

\section{Свойства}

\begin{enumerate}
    \item Неотрицательность: \(\xi \geq 0 \Rightarrow \mathbb{M}_\xi \geq 0\) по свойству математического ожидания

    \item \(\mathbb{D}_{a\cdot \xi} = a^2 \mathbb{D}_\xi\) - выносим а по определению математического ожидания

    \item Дисперсия суммы независимых случайных величин есть сумма их маргинальных дисперсий:
    \[\mathbb{D}_{(\xi + \eta)} = \mathbb{D}_\xi + \mathbb{D}_\eta\]
    \textbf{Доказательство:}
    \(\mathbb{D}_{(\xi + \eta)} = \mathbb{M}[(\xi + \eta - (\mathbb{M}[\xi + \eta])^2)] = \mathbb{M}[((\xi - \mathbb{M}_\xi) + (\eta - \mathbb{M}_\eta))^2] = \mathbb{M}[(\xi - \mathbb{M}_\xi)^2 + (\eta - \mathbb{M}_\eta)^2 + 2 (\xi - \mathbb{M}_\xi) (\eta - \mathbb{M}_\eta)] = \mathbb{D}_\xi + \mathbb{D}_\eta + 0 = \mathbb{D}_\xi + \mathbb{D}_\eta
    \)
    
    
    \rmk{Откуда там 0? \(\mathbb{M}[(\xi - \mathbb{M}_\xi)(\eta - \mathbb{M}_\eta)] = \mathbb{M}[\xi - \mathbb{M}_\xi] \cdot \mathbb{M}[\eta - \mathbb{M}_\eta] = 0^2 = 0\), потому что математическое ожидание от произведения независимых случайных величин есть произведение маргинальных математических ожиданий}

    \item Для зависимых случайных величин \(\mathbb{D}[\xi + \eta]= \mathbb{D}_\xi + \mathbb{D}_\eta + 2cov(\xi\eta)\)
    
\end{enumerate}

\rmk{Квадрат делает величины меньше 1 еще меньше, а больше 1 - еще больше. Чтобы минимизировать дисперсию, используется метод наименьших квадратов}

\section{Среднее квадратическое отклонение}
\defn{Среднее квадратическое отклонение}{
Корень из дисперсии

\[\sigma = \sqrt{D_\xi}\]
}
\end{document}