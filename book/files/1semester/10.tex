\documentclass[../Main.tex]{subfiles}
\begin{document}
\chapter{Ковариация}

\section{Определение}
Появляется, когда случайные величины зависимы, и выражает степень зависимости между ними.

\defn{Ковариация}{
\(cov(\xi, \eta) = \mathbb{M}[(\xi - \mathbb{M}_\xi)(\eta - \mathbb{M}_\eta)]\)
}

\subsection{Для практики}
\(
cov(\xi, \eta) = \mathbb{M}[\xi, \eta] - \mathbb{M}_\xi \mathbb{M}_\eta - \mathbb{M}_\xi \mathbb{M}_\eta + \mathbb{M}_\xi \mathbb{M}_\eta = \mathbb{M}[\xi, \eta] - \mathbb{M}_\xi \mathbb{M}_\eta
\)

\section{Некоррелированность}

\fact{Если случайные величины независимы, то их ковариация равна 0. Но нулевая ковариация не означает независимость случайных величин}

\fact{Для зависимых \(\xi, \eta\) выполняется следующее:
\[\mathbb{D}[\xi + \eta] = \mathbb{D}_\xi + \mathbb{D}_\eta + 2cov(\xi, \eta)\]}

\defn{Некорреллированность}{
Случайные величины, ковариация которых равно 0, называются некоррелированными. Из независимости следует некоррелированность, но из некоррелированности не следует независимость.
}

\subsection{Доказательство}

Пусть \(\alpha: \{0, \frac{\pi}{2}, \pi\}\) --- равномерно распределённая случайная величина, принимающая значения с равной вероятностью \(\frac{1}{3}\). Определим случайные величины \(\xi = \sin(\alpha)\) и \(\eta = \cos(\alpha)\), которые также равномерно распределены.

Значения случайных величин \(\xi\) и \(\eta\) для каждого \(\alpha\) приведены в таблице:

\begin{center}
    \begin{tabular}{|c|c|c|c|}
        \hline
        \(\alpha\) & \(\xi\) & \(\eta\) & \(\mathbb{P}\) \\ 
        \hline
        0 & 0 & 1 & \(\frac{1}{3}\) \\ 
        \hline
        \(\frac{\pi}{2}\) & 1 & 0 & \(\frac{1}{3}\) \\  
        \hline
        \(\pi\) & 0 & -1 & \(\frac{1}{3}\) \\ 
        \hline
    \end{tabular}
\end{center}

Известно, что \(\xi\) и \(\eta\) связаны основным тригонометрическим тождеством: \(\xi^2 + \eta^2 = \sin^2(\alpha) + \cos^2(\alpha) = 1\).

Вычислим ковариацию случайных величин \(\xi\) и \(\eta\):

\[
\text{cov}(\xi, \eta) = \mathbb{M}[\xi \eta] - \mathbb{M}[\xi] \cdot \mathbb{M}[\eta].
\]

\(\mathbb{M}[\xi]\):
\[
\mathbb{M}[\xi] = \sum \xi_i \cdot \mathbb{P}(\alpha_i) = 0 \cdot \frac{1}{3} + 1 \cdot \frac{1}{3} + 0 \cdot \frac{1}{3} = 0 + \frac{1}{3} + 0 = \frac{1}{3}.
\]

\(\mathbb{E}[\eta]\):
\[
\mathbb{M}[\eta] = \sum \eta_i \cdot \mathbb{P}(\alpha_i) = 1 \cdot \frac{1}{3} + 0 \cdot \frac{1}{3} + (-1) \cdot \frac{1}{3} = \frac{1}{3} + 0 - \frac{1}{3} = 0.
\]

\(\mathbb{M}[\xi \eta]\):
\[
\xi \eta =
\begin{cases} 
0 \cdot 1 = 0, & \text{если } \alpha = 0, \\
1 \cdot 0 = 0, & \text{если } \alpha = \frac{\pi}{2}, \\
0 \cdot (-1) = 0, & \text{если } \alpha = \pi.
\end{cases}
\]
\[
\mathbb{M}[\xi \eta] = \sum (\xi_i \eta_i) \cdot \mathbb{P}(\alpha_i) = 0 \cdot \frac{1}{3} + 0 \cdot \frac{1}{3} + 0 \cdot \frac{1}{3} = 0.
\]
\[
cov(\xi, \eta) = \mathbb{M}[\xi \eta] - \mathbb{M}[\xi] \cdot \mathbb{M}[\eta] = 0 - \frac{1}{3} \cdot 0 = 0.
\]

Таким образом, ковариация равна:
\[
\text{cov}(\xi, \eta) = 0.
\]

\rmk{Несмотря на то, что \(\xi\) и \(\eta\) связаны тождеством \(\xi^2 + \eta^2 = 1\), их ковариация равна нулю, что указывает на отсутствие линейной зависимости. Однако это не означает, что \(\xi\) и \(\eta\) независимы, так как они зависимы стохастически и функционально.}

\end{document}