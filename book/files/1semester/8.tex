\documentclass[../Main.tex]{subfiles}
\begin{document}
\chapter{Плотность}

\rmk{Небольшой экскурс в прошлое: вспомним, что мы знаем о случайных величинах:
\begin{itemize}
    \item Есть дискретные случайные величины, они задаются законом распределения: множеством значений и вероятностью их получения, а также функцией распределения - вероятностью, что случайная величина примет значение, чем x.
    \item Есть непрерывные случайные величины, вместо закона распределения у них плотность - это производная функции распределения.
\end{itemize}

А теперь формально:
}

\section{Определение}

\defn{Плотность}{
Функция, которая описывает вероятность того, что случайная величина примет определённое значение в заданном интервале.

Причем такая функция, что интеграл \(\overset{+ \infty}{\underset{- \infty}{\int}}\rho_\xi(\tau)d\tau = \mathbb{F}_\xi(x)\).
}

\section{Свойства}
\begin{enumerate}
    \item Интеграл от плотности равен 1: \(\overset{+ \infty}{\underset{- \infty}{\int}}\rho_\xi(x)dx = 1\)
    \item Неотрицательность: \(\rho_\xi(x) \geq 0 \ \forall \ x \in (-\infty, \ +\infty)\)

    \textbf{Доказательство:} Функция распределения является неубывающей — она либо возрастает, либо остаётся постоянной. Значит, её производная (плотность вероятности) всегда будет величиной неотрицательной.

    \item Вероятность принадлежности промежутку \(<x_1, x_2>\) равна \(\overset{x_2}{\underset{x_1}{\int}}\rho_\xi(x)dx\).

    \textbf{Доказательство:} \(\mathbb{P}(x_1 \leq \xi < x_2) = \mathbb{F}_\xi(x_2) - \mathbb{F}_\xi(x_1) = \overset{x_2}{\underset{- \infty}{\int}}\rho_\xi(x)dx - \overset{x_1}{\underset{- \infty}{\int}}\rho_\xi(x)dx = \overset{x_1}{\underset{- \infty}{\int}}\rho_\xi(x)dx \ + \ \overset{x_2}{\underset{x_1}{\int}}\rho_\xi(x)dx - \overset{x_1}{\underset{- \infty}{\int}}\rho_\xi(x)dx = \overset{x_2}{\underset{x_1}{\int}}\rho_\xi(x)dx \)
\end{enumerate}

\rmkb{
Заметим, что плотность определена неоднозначно. А вероятность, что случайная величина примет конкретное значение, равна 0.
}

\pf{
Если у нас есть \(\xi\), заданная по показательному закону, то

\[\rho_\xi = \begin{cases}
    0, x < 0 \\ \lambda e^{-\lambda x}, x \geq 0
\end{cases} \ \mathbb{P}_\xi (0) = \lambda
\]

\[\rho_{1\xi} = \begin{cases}
    0, x \leq 0 \\ \lambda e^{-\lambda x}, x > 0
\end{cases} \ \mathbb{P}_{1\xi}(0) = 0
\]

Это множество меры 0, так что вероятность определена неоднозначно :(
}

\end{document}