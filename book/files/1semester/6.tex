\documentclass[../Main.tex]{subfiles}
\begin{document}
\chapter{(Абсолютно) непрерывная случайная величина}
\(\xi,\ \mathbb{F}_\xi(x)\ \exists\ \rho(x)\ -\) функция-плотность, такая, что \(\mathbb{F}_\xi(x) = \overset{+\infty}{\underset{-\infty}{\int}}\rho(\tau)d\tau\), при этом \(\mathbb{F}_\xi\) должна быть дифференцируемой.

Как найти плотность? \(\rho(x) = \mathbb{F}_\xi'(x)\).

\defn{Непрерывная случайная величина}{
Случайная величина, у которой существует плотность, интеграл которой равен функции распределения. Задается плотностью (аналог закона распределения дискретной случайной величины) и функцией распределения.
}

\exm{1}{
\(\xi\) равномерно распределена на отрезке \([a, b]\). 
\[\rho = \begin{cases} 0, x \notin [a,\ b]\\ \dfrac{1}{b-a}, x \in [a,\ b] \end{cases}\]

\(\mathbb{F}_\xi(x) =\)

\(x < a: \overset{+\infty}{\underset{-\infty}{\int}}0 d\tau = 0\)

\(a \leq x \leq b: \overset{a}{\underset{-\infty}{\int}} \rho(\tau) d\tau + \overset{x}{\underset{a}{\int}}\rho(\tau)d\tau = 0+\overset{x}{\underset{a}{\int}}\dfrac{d\tau}{b-a} = \dfrac{\tau}{b-a}|^x_a = \dfrac{x-a}{b-a}\)

\(x > b: \overset{a}{\underset{-\infty}{\int}}\rho(\tau)d\tau + \overset{b}{\underset{a}{\int}}\rho(\tau)d\tau + \overset{x}{\underset{b}{\int}}\rho(\tau)d\tau = 0+ \dfrac{\tau}{b-a}|^b_a+0=\dfrac{b-a}{b-a}=1\)
}
\rmk{b и a можно включать куда угодно, потому что вероятность попадания в конкретную точку = 0.}

\rmkb{Могут существовать точки, где плотности не существует.}

\exm{2. Показательное распределение}{
\(\xi, \lambda > 0\)

\[\mathbb{P}_\xi(x)=\begin{cases} 0, x<0\\ \lambda e^{-\lambda x}, x \geq 0 \end{cases}\]

\(\xi ~ \mathbb{P}(\lambda)\)

Про память у распределения:

\(A = \{ \xi > t \}, B = \{\xi > t + \tau \},\ B > A\)

\(\mathbb{P}(B/_A) = \dfrac{\mathbb{P}(BA)}{\mathbb{P}(A)} = \dfrac{\mathbb{P}(B)}{\mathbb{P}(A)}\)

\[\mathbb{F}_\xi = \begin{cases} 0,\ x < 0\\ \overset{0}{\underset{-\infty}{\int}}\lambda e^{-\lambda \tau}d\tau + \overset{x}{\underset{0}{\int}} \lambda e^{-\lambda \tau} d\tau = 0 + \overset{x}{\underset{0}{\int}} \lambda e^{-\lambda \tau} d\tau = -e^{-\lambda \tau}|^x_0 = 1- e^{-\lambda x}, x \geq 0 \end{cases}\]

\(\mathbb{P}(\xi > x) = 1 - 1 + e^{-\lambda x} = e^{-\lambda x}\)

\(\mathbb{P}(B/_A) = \dfrac{e^{-\lambda (t + \tau)}}{e^{-\lambda t}} = e^{-\lambda \tau} = \mathbb{P}(\xi > \tau)\)
}

\exm{3. Гауссовский закон (нормальное распределение)}{
\(\xi\) эквивалентно \(N(a, \sigma), -\infty < a < + \infty, \sigma 
> 0\)

\(\rho (x) = \dfrac{1}{\sqrt{2 \pi}\sigma} e^{-\dfrac{(x-a)^2}{2 \sigma^2}}\)

\(\mathbb{F}_\xi = \overset{x}{\underset{-\infty}{\int}}e^{-\dfrac{(-\tau - a)^2}{2 \sigma^2}}d\tau\)
}
\end{document}