\documentclass[../Main.tex]{subfiles}
\usepackage{tabularx}
\usepackage{amssymb}

\begin{document}
\chapter{Урновая система}

\rmkb{Есть сочетания и размещения. Чтобы их не перепутать, нужно запомнить, что в размещениях (от слова место) важен порядок, а в сочетаниях (нет слова место) - порядок не важен. А вот уже есть повторения или нет - надо думать.}

\section{Схема выбора, приводящая к сочетанию без повторений}
\defn{Сочетание без повторений}{
Если опыт состоит в выборе m элементов без возвращения и упорядочивания из n-элементного множества, то различными исходами будет число \(C_n^m = \dfrac{n!}{m!(n-m)!}\)
}

\exm{1}{ \\
В урне 10 шаров, среди них есть черный. Какова вероятность, что его вытащили?

Общая вероятность этого события:

\(P(A) = \dfrac{C_9^2}{C_{10}^3} \)

Мы вытащили 3 шара, среди них уже есть черный. Поэтому из оставшихся 9 нужно выбрать 2 шара. А всего из 10 нужно выбрать 3.
}

\section{Схема выбора, приводящая к размещению без повторений}
\defn{Размещение без повторений}{
Если опыт состоит в выборе m элементов без возвращения, но с упорядочиванием из n-элементного множества, то различными исходами будет число \(A_n^m = \dfrac{n!}{(n-m)!}\)
}

\exm{2}{ \\
В урне 10 шаров, среди них есть черный. Какова вероятность, что его вытащили последним?

Общая вероятность этого события:

\(P(A) = \dfrac{A_9^2}{A_{10}^3} \)

Мы вытащили 3 шара, среди них черный уже последний. Поэтому из оставшихся 9 нужно выбрать 2 шара. А всего из 10 нужно выбрать 3.
}

\rmk{Числитель и знаменатель считаются по одной и той же схеме!}

\section{Схема выбора, приводящая к сочетанию с повторением}
\defn{Сочетание с повторением}{
Если опыт состоит в выборе m элементов с возвращением, но без упорядочивания из n-элементного множества, то различными исходами будет число \(C_{n+m-1}^m\)
}

\exm{3}{ \\
В урне 10 шаров разных цветов. Шар вытаскивается, смотрится его цвет и возвращается обратно. Какова вероятность, что все шары разного цвета?

Общая вероятность этого события:

\(P(A) = \dfrac{C_10^4}{C_{10+4-1}^4} \)

Мы вытащили 4 шара, среди них уже есть черный. Поэтому из оставшихся 9 нужно выбрать 2 шара. А всего из 10 нужно выбрать 3.
}

\section{Схема выбора, приводящая к размещению с повторением}
\defn{Размещение с повторениями}{
Если опыт состоит в выборе m элементов без возвращения и упорядочивания из n-элементного множества, то различными исходами будет число \(n^m\)
}

\exm{4}{ \\
В урне 5 шаров: серый, лазуревый, оранжевый, голубой, зеленый.

Общая вероятность этого события:

\(P(A) = \dfrac{1}{5^4} \)

Переставленные буквы - это другое слово ;)
}

\begin{center}
    \begin{tabular}{|l|l|l|}
        \hline
        & Без упорядочивания & С упорядочиванием \\ 
        \hline
        Без возвращения & \(C_n^m\) & \(A_n^m\) \\ 
        \hline
        С возвращением  & \(C_{n+m-1}^m\) & \(n^m\) \\ 
        \hline
    \end{tabular}
\end{center}

\newpage
\chapter{Алгебра}

\section{Понятие алгебры}
\(] \ \exists \ \Omega = \{\omega_1, \omega_2, \dots, \omega_n, \dotsc\} \ -\) набор элементарных исходов.
\(\mathfrak{A} -\) событие = выделенный набор подмножеств, который может не совпадать со всем множеством.

\defn{Алгебра}{
Набор подмножеств называется алгеброй, если в нём есть невозможное и достоверное событие, а также определены и существуют суммы, произведения и разности для любых двух элементов подмножества.
}

\defn{Сигма-алгебра}{
Алгебра называется сигма-алгеброй, если объединение и пересечение любого счетного набора \(\{A_i\}, i \in I\), где I - счетный набор индексов, принадлежит данной алгебре.
}

\rmkb{Определение сигма-алгебры избыточно}

\defn{Случайное событие}{
Элемент множества \(\mathfrak{a}\) 
}

Для \(\forall A \in \mathfrak{a}\) определена \(P(A)\) (вероятностная мера - числовая функция), удовлетворяющая следующим 4 аксиомам:
\begin{enumerate}
    \item Аксиома неотрицательности: \(P(A) >= 0\).
    \item Аксиома нормирования: вероятность достоверного события равна 1 (\(P(\Omega) = 1\)).
    \item Аксиома счетной аддитивности: \(] \{A_i\}, i \in I, \in \mathfrak{a}. P(\bigcup_{i \in I} A_i) = \underset{i \in I}{\sum} P(A_i)\). Вероятность объединения - сумма вероятностей.
    \item Аксиома полноты: \(] \exists A \in \mathfrak{a}. \ P(A) = 0 \rightarrow \forall B \subset A: P(B) = 0\). 
    
    Если вероятность события А = 0 , то вероятность определена 0 для всех подмножеств А, при этом \(P\) - вероятностная мера, а тройка \(\Omega, \ \mathfrak{a},\ \mathbb{P} \) - вероятностное пространство.
\end{enumerate}

\defn{Вероятностное пространство}{
Множество элементарных исходов, построенная на нем сигма-алгебра и определенная на ней вероятностная мера.
}

\section{Основные соотношения между вероятностными событиями}
\begin{enumerate}
    \item Противоположное событие \(\overline A\): сумма с исходным = достоверное, произведение с исходным - невозможное, его вероятность = \(1 - P(A)\). 
    
    \pf {\(P(\Omega) = P(A + \overline{A}) = P(A) + P(\overline A) = 1\)}
    \item \(]\ A \subset B, P(A) <= P(B),\ P(B) = P(A) + P(B \setminus A) \geq 0 \).
    
    Используя аксиомы счетной аддитивности и неотрицательности можно считать, что вероятность суммы равна сумма вероятностей.
    \item Вероятность события А не превышает 1. (\(P(A) \leq 1\)).

    \pf {\(A \subset \Omega \rightarrow P(A) (2\ast) \leq P(\Omega)\)}
    \item Вероятность невозможного события = 0. (\(P(\O) = 0\)).

    \pf{ 
    \(\overline{\O} = \Omega \rightarrow P(\overline{\O}) = P(\Omega) \rightarrow P(\O) = 1 - P(\Omega) = 1 - 1 = 0\).
    }
    \item Про взаимодействие двух событий:
        \begin{enumerate}
            \item \(P(A \setminus B) = P(A) - P(AB)\)
            \item \(P(B \setminus A) = P(B) - P(AB)\)
            \item \(P(A+B)=P(A)+P(B) - P(AB)\)
        \end{enumerate}
    \pf{
    \begin{enumerate}
        \item \(A=A\setminus_ B + AB\), тогда \(P(A)=P(A\setminus B) + P(AB)\). Выразим \(P(A\setminus B) = P(A)-P(AB)\), ч.т.д.
        \item \(B=B\setminus_ A + BA\), тогда \(P(B)=P(B\setminus A) + P(AB)\). Выразим \(P(B\setminus A) = P(B)-P(AB)\), ч.т.д.
        \item \(A+B=A\setminus B + B\setminus A+AB\), они несовместны между собой. По аксиоме счетной аддитивности \(P(A+B) = P(A \setminus B) + P(B \setminus A) + P(AB)\)
    \end{enumerate}
    }
    \item Формула сложения для n слагаемых: 
    
    \(A_1, \ A_2, \dotsc, \ A_n \)

    \(P_1 = \underset{i = 1}{\overset{n}{\sum}}P(A_i)\)

    \(P_2 = \underset{\ 1 \leq i < j \leq n}{\overset{n}{\sum}}P(A_i A_j)\)

    \(\dotsc\)
    
    \(P_n=(A_1 \cdot \dotsc \cdot A_n)\)

    \(P(A_1+\dotsc+A_n) = P_1 - P_2 + \dotsc \pm P_n\)

    \pf {Методом математической индукции:

    n = 2: по \(5^\circ\) \(P(A_1+A_2) = P(A_1) + P(A_2) - P(A_1A_2)\)

    n = k: \(P(\underset{i = 1}{\overset{k}{\sum}}A_i) = (\underset{i = 1}{\overset{k}{\sum}}P(A_i)) - (\underset{\ 1 \leq i < j \leq k}{\sum}P(A_i A_j)) + \dotsc \pm P(A_1 \cdot \dotsc \cdot A_k)\)

    n = k+1: \rmk{Дальше отсебятина, потому что оригинальное доказательство непонятно}
    Рассмотрим \(P(\underset{i = 1}{\overset{k+1}{\sum}}A_i) = P(\underset{i = 1}{\overset{k}{\sum}}A_i + A_{k+1})\). В предыдущем шаге мы предположили значение \(P(\underset{i = 1}{\overset{k}{\sum}}A_i) \). А теперь по свойству \(5^\circ:\) 
    \[ P(\underset{i = 1}{\overset{k+1}{\sum}}A_i) = P(\underset{i = 1}{\overset{k}{\sum}}A_i)+P(A_{k+1}) - P(\underset{i = 1}{\overset{k}{(\sum}}A_i) \cdot A_{k+1})\]
    \rmk{Начинаем мазохизм и страдание, вспоминаем включения-исключения из курса дискретной математики и подставляем в \(P(\underset{i = 1}{\overset{k}{(\sum}}A_i) \cdot A_{k+1})\):}

    Это будет: \((\underset{i = 1}{\overset{k}{\sum}}P(A_i \ \cdot \ A_{k+1})) - (\underset{\ 1 \leq i < j \leq k}{\sum}P(A_i A_j A_{k+1})) + \dotsc \pm P(A_1 \cdot \dotsc \cdot A_k \cdot A_{k+1})\)

    Теперь поставим: \((\underset{i = 1}{\overset{k}{\sum}}P(A_i)) - (\underset{\ 1 \leq i < j \leq k}{\sum}P(A_i A_j)) + \dotsc \pm P(A_1 \cdot \dotsc \cdot A_k) + P(A_{k+1}) - ((\underset{i = 1}{\overset{k}{\sum}}P(A_i \ \cdot \ A_{k+1})) - (\underset{\ 1 \leq i < j \leq k}{\sum}P(A_i A_j A_{k+1})) + \dotsc \pm P(A_1 \cdot \dotsc \cdot A_k \cdot A_{k+1}))\). Когда мы это упростим и сгруппируем, то получим ту формулу, которую нам надо =)
    }
\end{enumerate}

\end{document}